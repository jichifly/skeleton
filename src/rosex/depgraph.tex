% readme.tex
% 8/16/2011 jichi
%
% README for usage of dependence graph.

%\documentclass[11pt]{article}
\documentclass[preprint]{llncs}
%\documentclass[11pt]{llncs}
%\usepackage[silent]{TODO}
\usepackage{listings}
\usepackage{verbatim}
\usepackage{makeidx}
\usepackage{appendix}
\usepackage{graphicx}   % for importing graph
%\usepackage{natbib}     % cite in author-year style
%\usepackage{ifthen}

%\usepackage[dvips,
%  pdfpagemode=UseOutlines,bookmarks,
%  pdfstartview=FitH,
%  pdftitle={Proposal for exploiting OpenMP task-level parallelization},
%  pdfauthor={Jichi Guo},
%  colorlinks,linkcolor=blue,citecolor=blue,urlcolor=red
%  ]{hyperref}

%% Page setup
%\textwidth 6.5in
%\textheight 8.8in
%\oddsidemargin 0in
%\evensidemargin 0in
%\topmargin -0.3in
%\parindent 0.25in

\makeindex

\title{ROSE Dependency Graph}
\author{Jichi Guo \\ jguo@cs.utsa.edu}
\institute{University of Texas at San Antonio}

\begin{document}
\maketitle

%%
%  Structure 
%  - Introduction
%  - Input and output
%  - Vertices and edges
%  - Build options
%

%%
%  Section  Introduction
%
\section{Introduction}
\label{sec:intro}

This document serves to describe the usage of the \texttt{DepGraph} class in ROSE.
The following sections will introduce
  how to build the dependency graph (Section~\ref{sec:build}),
  how to traverse the graph  (Section~\ref{sec:traverse}),
  and advanced building options (Section~\ref{sec:option}).

%%
%  Section  Input and output
%  - Input
%  - Output
%
\section{Build dependency graph}
\label{sec:build}

\subsection{Input}
\label{sec:build:input}
The dependency graph is represented by \texttt{DepGraph} built from a Sage AST.
The AST is supposed to be either \texttt{SgFunctionDefinition} or \texttt{SgStatement}.
Additional, the advanced building options
  such as side effect interface or alias interface
  could be specified through the \texttt{option\_type}.
The advanced options are discussed later in Section~\ref{sec:option}.
An example to given in List~\ref{list:build}.
The graph instance is not \emph{valid} unless being built successfully.

\lstset{
  caption={Build \texttt{DepGraph}},
  label=list:build,
  language=C++,
  showstringspaces=false,
  basicstyle=\footnotesize,
  captionpos=b}
\begin{lstlisting}
  DepGraph g;
  g.build(ast_root);
  if (!g.valid())
    cerr << "Failed to build the dependency graph." << endl;
\end{lstlisting}

\subsection{Output}
\label{sec:build:output}
The dependency graph is represented as a directed graph.
The vertex represents the corresponding statement in the AST.
There is a one-to-one relation between the graph vertex and the sage statement.
The edge represents the dependency between statements.
The detailed dependency information is stored as \texttt{DepInfo} in the edge.

\subsection{Copy}
\label{sec:build:copy}
The internal graph representation is managed using \texttt{boost::shared\_ptr}.
When copying a graph instance, a vertex, or an edge,
  the underlying representation will not get duplicated
  and all instances will share the same implementation.
So, they can be viewed as the descriptor of the underlying implementation.
An example is explained in List~\ref{list:copy}.
The underlying implementation will not be deleted unless
  it is not referenced anymore.

\lstset{
  caption={Copy \texttt{DepGraph}},
  label=list:copy,
  language=C++,
  showstringspaces=false,
  basicstyle=\footnotesize,
  captionpos=b}
\begin{lstlisting}
DepGraph g1(ast_root);
ROSE_ASSERT(g1);    // Assert g1.valid()
DepGraph g2 = g1;   // Now g1 and g2 shared the same implementation.
g1.clear();         // g1's pointer to the implementation is reset to 0.
                    // However, g2 is not influenced.
DepGraph::vertex_type v1 = g.entry(); // Get the first vertex in the graph.
DepGraph::vertex_type v2 = v1; // Now, v1 and v2 share the same impl.
DepGraph::vertex_type v3;
v3.reset(v1.get()); // Set v3's pointer to v1's. Same as v3 = v1.
\end{lstlisting}

%%
%  Section  Vertices and edges
%  - Properties
%  - Iteration
%
\section{Iterate the dependency graph}
\label{sec:traverse}

The dependency graph could be iterated using similar way to \emph{Boost Graph Library}.
An example that traverse and dump the graph to \texttt{std::cout} is illustrated in List~\ref{list:iteration},
  which is equivalent to invoke \texttt{dump()} function.

\lstset{
  caption={Iterate \texttt{DepGraph}},
  label=list:iteration,
  language=C++,
  showstringspaces=false,
  basicstyle=\footnotesize,
  captionpos=b}
\begin{lstlisting}
  DepGraph g(ast_root);
  if (g) {
    DepGraph::vertex_type v;
    DepGraph::edge_type e;
    BOOST_FORALL_VERTICES (v, g, typeof(g)) {
      cout << "vertex: " << v << endl;
      BOOST_FORALL_OUTEDGES (v, e, g, typeof(g)) {
        cout << "edge: " << e << endl;
      }
    }
  }
\end{lstlisting}

The \texttt{toDOT()} method will generate a DOT file for the graph.
The \texttt{DepGraphPropertiesWriter} class in the source file
  is an example to use \texttt{boost::write\_graphviz} to generate customized DOT file.

Additionally, the graph could be converted to \texttt{boost::adjacency\_list}
  using \texttt{to\_boost\_adjacency\_list()}.
Such conversion is also illustrated in List~\ref{list:boost}.

\lstset{
  caption={\texttt{DepGraph} to \texttt{boost::adjacency\_list} conversion},
  label=list:boost,
  language=C++,
  showstringspaces=false,
  basicstyle=\footnotesize,
  captionpos=b}
\begin{lstlisting}
  DepGraph g(ast_root); // Dependency graph to convert
  ROSE_ASSERT(g);

  typedef DepGraph G;
  typedef adjacency_list<
    vecS, vecS, bidirectionalS, G::vertex_type, G::edge_type
    > BG;

  BG bg(num_vertices(g)); // Target boost graph to build
  BOOST_AUTO(indices, get(vertex_index_t(), g);
  BOOST_FOREACH (G::vertex_type v, vertices(g)) {
    bg[indices[v]].reset(v.get()); // save g's vertex info to bg
    BOOST_FOREACH (G::edge_type e, out_edges(v, g)) {
      graph_traits<typeof(bg)>::edge_descriptor ed;
      bool succ;
      tie(ed, succ) = add_edge(indices[v], indices[target(e, g)], bg);
      if (succ)
        bg[ed].reset(e.get()); // save g's edge info to bg
    }
  }

\end{lstlisting}



%%
%  Section  Build options
%  - Side effect interface -- function/statement
%  - Alias interface
%  - Array interface
%
\section{Build options}
\label{sec:option}
The advanced building options
  such as side effect interface or alias interface
  could be specified through the \texttt{option\_type}.
An example enabling function side effect and induction variable substitution
  is given in List~\ref{list:option}.

\lstset{
  caption={Options to build \texttt{DepGraph}},
  label=list:option,
  language=C++,
  showstringspaces=false,
  basicstyle=\footnotesize,
  captionpos=b}
\begin{lstlisting}
  DepGraph g;
  DepGraph::option_type g_option;

  FunctionSideEffect funcInfo;
  IVSideEffectCollect stmtInfo;
  g_option.funcInfo = &funcInfo;
  g_option.stmtInfo = &stmtInfo;

  g.build(ast_root, g_option);
\end{lstlisting}

\subsection{Function memory side effects}
\label{sec:option:funcion}
By default, the statement with function calls will be marked as having IO dependency.
By specifying the \texttt{stmtInfo} in building option as \texttt{FunctionSideEffect},
  the dependency graph will be able to look into the function definition
  to compute memory its side effects
  (showed in previous example List~\ref{list:option}).

\subsection{Induction variable substitution}
\label{sec:option:ivs}
\emph{Induction variable substitution} (\emph{IVS}) is a transformation which can significantly improve
 the accuracy of dependency graph built for iterations such as loops.
To enable such substitution,
  the substitutions are needed to be computed and attached to the corresponding AST node
  before building dependency graph.
Then adding \texttt{IVSideEffectCollect} to statement side effect interface in the building option
  will enable the graph builder to use attached attributes to compute statement side effects
  and hence get a more accurate dependency graph.
An example is given in List~\ref{list:ivs} showing how to build a dependency graph
  with \emph{IVS} support.

\lstset{
  caption={Enabling IVS},
  label=list:ivs,
  language=C++,
  showstringspaces=false,
  basicstyle=\footnotesize,
  captionpos=b}
\begin{lstlisting}
  IVAttribute::addToFunction(sage_function);    // Compute attributes.
  IVAttribute::dump(sage_function); // Output all attributes to cout.

  DepGraph g;
  DepGraph::option_type g_option;

  IVSideEffectCollect sideEffect;
  g_option.stmtInfo = &sideEffect;

  g.build(sage_function, g_option);

  IVAttribute::removeAll(sage_function);    // Delete all added attributes.
\end{lstlisting}

\end{document}

% EOF %
